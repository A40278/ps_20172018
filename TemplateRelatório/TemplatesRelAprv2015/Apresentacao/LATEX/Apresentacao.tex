
\documentclass{beamer}
% 
%\usepackage[utf8]{inputenc}
\usepackage{array}
\usepackage[absolute,overlay]{textpos}
\usepackage{rotating}
%\usepackage{mathptmx}
\usepackage{graphicx}

% Packages a utilizar e respetivos par�metros.
\usepackage[latin1]{inputenc}
\usepackage[portuguese]{babel}

%\usepackage{helvet}
%\usepackage{extsize}

\usepackage{algorithm}
\usepackage{algorithmic}
\renewcommand{\algorithmicrequire}{\textbf{Dados:}}
\renewcommand{\algorithmicensure}{\textbf{Resultado:}}

\newcolumntype{x}[1]{%
>{\centering}p{#1}}%


\usefont{T1}{ptm}{m}{n}
\selectfont

\setlength{\TPHorizModule}{30mm} 
\setlength{\TPVertModule}{\TPHorizModule} 
\setlength{\parindent}{0pt} 
%\TPGrid[40mm,40mm]{15}{25}  % 3 - 1 - 7 - 1 - 3 Columns
%\TPGrid{210}{297} 

\usetheme[compress]{Singapore}
%\usetheme[compress]{Pittsburgh}
%\usetheme[compress]{PaloAlto}


\setbeamertemplate{footline}[frame number]
\setbeamercovered{transparent}
\beamertemplatenavigationsymbolsempty
\setbeamertemplate{navigation symbols}{}


%\setcounter{tocdepth}{4} 
%
% 0
%
%%% title page
\title{\textsc{*T\'{i}tulo do Trabalho de Projecto*}}
%\subtitle{}
\author{
	{\large \textbf{*Fernando Pessoa*\\ *Ricardo Reis*}} \\
}

\vspace{2cm} 

\institute
{
	{\normalsize Licenciatura em Engenharia Inform\'{a}tica e de Computadores}
	

	\vspace{.1cm}
  {\normalsize Projecto e Semin\'{a}rio} \\ 
  \vspace{.5cm}
  {\normalsize 
	\begin{tabular}{ll}
  {Orientadores:} & *\'{A}lvaro de Campos* \\
                  & *Alberto Caeiro, SoftCompany*\\
  \end{tabular}}\\


}
\vfill
\date{Apresenta\c{c}\~ao  * *\\ Maio de 2015}


%%%
\begin{document}

%
% Slide 1 - Capa
%% title
\begin{frame}[t,plain]
		%\begin{minipage}{14em}\mdseries
		\includegraphics[width=10em,height=6em]{logo_principal} 
		%\end{minipage}
\titlepage
\end{frame}

%
% Slide 2 - TOC
%
%% toc
\begin{frame}
\frametitle{Sum\'{a}rio}
\scriptsize{\tableofcontents}
\end{frame}

%
% Sec��o 1 - Introdu��o
%
\section[Introdu\c{c}\~ao]{Introdu\c{c}\~ao}

%
% Slide 3 - 
%
\begin{frame}{Introdu\c{c}\~ao}
Cen\'arios t\'ipicos de manipula\c{c}\~ao de dados (figura)
\begin{figure}
	\centering
	\includegraphics*[width=10cm,height=4.5cm]{cenario.png}
\end{figure}

\vfill
Os dados (lista enumerada):
\begin{itemize}
	\item organizados em $n$ padr\~oes, exemplos ou inst\^ancias
	\item cada padr\~ao tem $d$ caracter\'isticas (dimensionalidade)
	\item podem ser manipulados com diferentes prop\'ositos/objetivos
\end{itemize}
\end{frame}

%
% Sec��o 2 - O Problema
%
\section[O Problema]{O Problema}

%
% Slide 4 - 
%
\begin{frame}{O Problema: conceitos}
Exemplos de dados de baixa dimensionalidade e \textcolor{blue}{com AD} (tabela):
\begin{itemize}
	\item $d$ dimens\~oes, $c$ classes, $n$ padr\~{o}es 
	\item em baixa dimensionalidade, $n > d$ ou $n \gg d$
	\item com AD, tipicamente tem-se $d \gg n$
\end{itemize}

\vfill
\begin{table}[h!]
	\centering
	\scriptsize
	\scalebox{0.9}{
	\begin{tabular}{|l|r|r|r|r|l|r|r|r|r|}
		\hline
		\textbf{Conjunto}       & $d$   &  $c$   & $n$ &    \textbf{Tipo de Dados} & \textbf{Problema} \\ \hline 
    Car                     & 6     &   4    & 1728 & Autom\'oveis & Possibilidade de Compra   \\ \hline
 	  Pima                    & 8     &   2    &   768   & Cl\'inicos & Dete\c{c}\~ao da Diabetes \\ \hline 
    Contraceptive           & 9     &   2    &  1473   & Cl\'inicos & M\'etodo de Contrace\c{c}\~ao \\ \hline
    Wine                    & 13    &   3    &  178 & Qu\'imicos & Tipo de Vinho \\ \hline
    Hepatitis               & 19    &   2    &  155 & Cl\'inicos & Dete\c{c}\~ao  de Hepatite \\ \hline 
    Dermatology             & 34    &   6    & 358 & Cl\'inicos & Doen\c{c}a de Pele \\ \hline \hline
    \textcolor{blue}{Colon}             & \textcolor{blue}{2000}    &   \textcolor{blue}{2}    &   \textcolor{blue}{62}   &    \textcolor{blue}{Exp. Gen\'etica} & \textcolor{blue}{Dete\c{c}\~ao de Cancro}\\ \hline 
\textcolor{blue}{SRBCT}   & \textcolor{blue}{2309}    &   \textcolor{blue}{4}    &   \textcolor{blue}{83}   &    \textcolor{blue}{Exp. Gen\'etica} & \textcolor{blue}{Dete\c{c}\~ao de Cancro} \\ \hline
	  \textcolor{blue}{TOX-171}	 				 & \textcolor{blue}{5748}	   &   \textcolor{blue}{4}	  &  \textcolor{blue}{171}	 &    \textcolor{blue}{Exp. Gen\'etica} & \textcolor{blue}{Dete\c{c}\~ao de Cancro} \\ \hline			 	
		\textcolor{blue}{Example1}         & \textcolor{blue}{9947}    &   \textcolor{blue}{2}    &  \textcolor{blue}{2000}   &    \textcolor{blue}{Texto} & \textcolor{blue}{Assunto da Not\'icia}       \\ \hline
		\textcolor{blue}{ORL10P}           & \textcolor{blue}{10304}   &  \textcolor{blue}{10}   &  \textcolor{blue}{100}    &    \textcolor{blue}{Faces} & \textcolor{blue}{Identifica\c{c}\~ao}  \\ \hline				
		\textcolor{blue}{11-Tumors}        & \textcolor{blue}{12553}   &  \textcolor{blue}{11}    &  \textcolor{blue}{174}   &    \textcolor{blue}{Exp. Gen\'etica} & \textcolor{blue}{Dete\c{c}\~ao de Cancro}	\\ \hline
  	\textcolor{blue}{Lung-Cancer}      & \textcolor{blue}{12601}   &   \textcolor{blue}{5}    &  \textcolor{blue}{203}   &    \textcolor{blue}{Exp. Gen\'etica} & \textcolor{blue}{Dete\c{c}\~ao de Cancro}	\\ \hline   		
 	%	\textcolor{blue}{SMK-CAN-187}	     & \textcolor{blue}{19993}	 &   \textcolor{blue}{2}	  &  \textcolor{blue}{187}	 &    \textcolor{blue}{Exp. Gen\'etica} & \textcolor{blue}{Dete\c{c}\~ao de Cancro}	\\ \hline  	 		
		\textcolor{blue}{Dexter}           & \textcolor{blue}{20000}   &   \textcolor{blue}{2}    & \textcolor{blue}{2600}   &    \textcolor{blue}{Texto} & \textcolor{blue}{Assunto da Not\'icia}      \\ \hline
		\textcolor{blue}{GLI-85}	         & \textcolor{blue}{22283}	 &   \textcolor{blue}{2}	  &   \textcolor{blue}{85}   &    \textcolor{blue}{Exp. Gen\'etica} & \textcolor{blue}{Dete\c{c}\~ao de Cancro} \\ \hline
	  \textcolor{blue}{Dorothea}         & \textcolor{blue}{1000000} &   \textcolor{blue}{2}    & \textcolor{blue}{1950}   &    \textcolor{blue}{Cl\'inicos} & \textcolor{blue}{Dete\c{c}\~ao de Composto}\\ \hline	
		\end{tabular}}
\end{table}
\end{frame}

%
% Sec��o 3 - A Solu��o
%
\section[A Solu\c{c}\~ao]{A Solu\c{c}\~ao}

%
% Slide 5
%
\begin{frame}{Propostas para Sele\c{c}\~ao: Complexidade}
Em termos de complexidade, tem-se para o RFS
\begin{equation} \nonumber
	C_{RFS} = \underbrace{O(nd)}_{Relevancia} + \underbrace{O(d\log d)}_{Ordenacao}
\end{equation}
RRFS apresenta a complexidade adicional de calcular as semelhan\c{c}as entre pares
\begin{equation} \nonumber
	C_{RRFS} = \underbrace{O(nd)}_{Relevancia} + \underbrace{O(d\log d)}_{Ordenacao} + \underbrace{O(nm)}_{Redundancia}
\end{equation}

\vfill
\begin{itemize}
	\item RRFS \'e mais r\'apido do que outros filtros (FCBF e MRMR)
	\item Nalguns casos, RRFS \'e o mais r\'apido com menos erro
	\item CFS \'e o mais lento (inadequado para dados com AD)
	\item Medidas MAD e MM adequadas para todos os tipos de dados
	\item Medida AMGM \'e mais adequada para dados esparsos
\end{itemize}
\end{frame}


%
% Sec��o 4 - Grande Ideia 1
%
\section[Grande Ideia 1]{Grande Ideia 1}

%
% Slide 6 - 
% 
\begin{frame}{Representa\c{c}\~ao de Dados}
Assim, identifica-se a necessidade de:
\begin{itemize}
	\item encontrar novas formas de representa\c{c}\~ao dos dados
  \item representar de forma independente da tarefa a jusante
  \item facilitar a visualiza\c{c}\~ao e an\'alise de dados com AD
\end{itemize}

\vfill
As t\'ecnicas de \textbf{sele\c{c}\~ao} (\emph{feature selection} - FS) e 
\textbf{discretiza\c{c}\~ao} (\emph{feature discretization} - FD) realizam essa representa\c{c}\~ao

\vfill
\textcolor{blue}{$\bullet$} \textbf{Sele\c{c}\~ao}\\
$\rightarrow$ escolha de sub-conjuntos de caracter\'isticas adequados

\vfill
\textcolor{blue}{$\bullet$} \textbf{Discretiza\c{c}\~ao}\\ 
$\rightarrow$ representa\c{c}\~oes discretas de caracter\'isticas num\'ericas\\
$\rightarrow$ com informa\c{c}\~ao suficiente para aprendizagem\\
$\rightarrow$ ignora ru\'ido e flutua\c{c}\~oes irrelevantes
\end{frame}


%
% Sec��o 5 - Grande Ideia 2
%
\section[Grande Ideia 2]{Grande Ideia 2}



%
% Slide 7
%
\begin{frame}{Sele\c{c}\~ao: Taxonomia e Op\c{c}\~oes Tomadas}
Algoritmos de sele\c{c}\~ao s\~ao categorizados como: 
\begin{itemize}
	\item [i)] filtros (\emph{filters})
	\item [ii)] envolvimento (\emph{wrapper}) 
	\item [iii)] embebidos (\emph{embedded})
	\item [iv)] h\'ibridos (\emph{hybrid})	
\end{itemize}

\vfill
Escolha inicial $\rightarrow$ filtros n\~ao supervisionados e supervisionados

\vfill
\begin{itemize}
	\item Baixa complexidade, efici\^encia e interpretabilidade
	\item Independentes da tarefa de minera\c{c}\~ao de dados 
  \item Alguns filtros existentes s\~ao:
   	\begin{itemize} 
  	   \item ineficientes (tempo) em dados com AD\footnote{\footnotesize{CFS-Correlation-based Feature Selection, MRMR-Maximum Relevance Minimum Redundancy, RELIEF-Recursive Elimination of Features}} 
  		 \item sens\'iveis ao problema de elevado $d$, baixo $n$\footnote{\footnotesize{FCBF - Fast Correlation-Based Filter}}
  	\end{itemize}   
\end{itemize}
\end{frame}


%
% Slide 8
%
\begin{frame}{O algoritmo PS}
\begin{table}
\begin{algorithmic}[1]
{\scriptsize
\REQUIRE $X$: $n \times d$ matrix, $n$ patterns of a $d$-dimensional training set.\\
	\indent \hspace{.6cm} $m \ (\leq d)$:, maximum number of features to keep.\\
	\indent \hspace{.6cm} $M_S$: maximum allowed similarity between pairs of features.\\
	\ENSURE $\mathrm{FeatKeep}$: an $m'-$dimensional array (with  $m' \leq m$) containing the indexes of the selected features.\\
\indent \hspace{.5cm} $\widetilde{X}:$ $n \times m'$ matrix, reduced dimensional training set, with features sorted by decreasing relevance.\\
	\vspace{1mm} \hrule \vspace{1mm}
	\small
	\STATE Compute the relevance $r_i$ of each feature $X_i$, for $i \in \{1,\ldots,d\}$,
using a dispersion measure (MAD, MM, AMGM, or IQR).	
	\STATE Sort the features by decreasing order of $r_i$. Let $i_1,i_2,...,i_d$ be the resulting
permutation of $\{1,...,d\}$ ({\it i.e.}, $r_{i_1} \geq r_{i_2} \geq ... \geq r_{i_d}$).
	\STATE $\mathrm{FeatKeep}[1] = i_1$; prev=1; next=2;  
	\FOR {$f=2$ to $d$}
	  \STATE $s = S(X_{i_f},X_{i_{\mathrm{prev}}})$; 
		\IF {$s<M_S$}
			\STATE $\mathrm{FeatKeep[next]} = i_f$; $\widetilde{X}_{\mathrm{next}} = X_{i_f}$; prev = $i_f$; next = next + 1; 
		\ENDIF
		\IF {next=m}
			\STATE break; \hfill \COMMENT{/* We have m features. Break loop. */}
	  \ENDIF
	\ENDFOR	
}
\end{algorithmic}
\end{table}
\end{frame}


%
% Sec��o 6 - Resultados Obtidos
%
\section[Resultados Obtidos]{Resultados Obtidos}


%
% Slide 9
%
\begin{frame}{Avalia��o Experimental: Relev\^{a}ncia}
Algoritmo \emph{relevance FS} (RFS): 
\begin{itemize}
	\item guarda as caracter\'isticas com maior \emph{relev\^ancia}
	\item relev\^ancia medida pela \emph{dispers\~ao}
\end{itemize}

Para o caso n\~ao supervisionado $\rightarrow$ medidas de \emph{relev\^ancia} $@rel$:\\
\vfill
\textcolor{blue}{$\bullet$} \emph{Mean absolute difference} $\mbox{MAD}_i = \frac{1}{n} \sum_{j=1}^{n} \left| X_{ij}  - \overline{X}_i \right|$

\vfill
\textcolor{blue}{$\bullet$} \emph{Mean-median} $\mbox{MM}_i = |\overline{X}_i - \mathrm{median}{(X_i)}|$ 
%o valor absoluto da diferen\c{c}a entre a m\'edia e a mediana de $X_i$

\vfill
\textcolor{blue}{$\bullet$} \emph{Arithmetic mean geometric mean}
\vspace{-.65cm}
\begin{equation} \nonumber
	\mbox{AMGM}_i = \frac{1}{n} \sum_{j=1}^{n} \exp(X_{ij}) / \left( \exp \left( \sum_{j=1}^{n} X_{ij} \right) \right)^\frac{1}{n}
\end{equation}

\vfill
Crit\'erio de relev\^ancia cumulativa $\sum_{f=1}^m r_{i_f} / \sum_{i=1}^d r_i = c_m / c_d \geq L,$ $L \in [0,8; 0,95] \rightarrow$ escolha do n\'umero de caracter\'isticas $m \ (< d)$
\end{frame}


%
% Slide 10
%
\begin{frame}{Propostas para Sele\c{c}\~ao: Resultados Experimentais}

{
\small
\textcolor{blue}{$\bullet$} 
Classifica\c{c}\~ao supervisionada (SVM linear), valida\c{c}\~ao cruzada (\emph{10-fold})\\
\textcolor{blue}{$\bullet$} 
Percentagem de erro de generaliza\c{c}\~ao com filtros \textcolor{blue}{supervisionados}\\
}

\begin{table}[h!]
 % \vspace{-0.25cm}
	\centering
	\footnotesize
	\scalebox{0.9}{
	\begin{tabular}{l|r|r|r|r|r|r|r|r|r|}
		\cline{2-10}
		                          & \multicolumn{3}{|c|}{\textbf{RRFS, $M_S=0.8$}} & \multicolumn{5}{|c|}{\textbf{Filtros Supervisionados\footnote{\footnotesize{Mutual Information (MI), RELIEF (RF), Correlation-based Feature Selection (CFS), Fast Correlation-based Filter (FCBF), Fishers Ratio (FiR) e
		                          Maximum Relevance Minimum Redundancy (MRMR)}}}} & \multicolumn{1}{|c|}{\textbf{Base}}\\ \hline
		\multicolumn{1}{|l|}{\textbf{Conjunto}} & \multicolumn{1}{|c|}{\textbf{MM}}  & \multicolumn{1}{|c|}{\textbf{FiR}} & \multicolumn{1}{|c|}{\textbf{MI}} & \multicolumn{1}{|c|}{\textbf{RF}} & \multicolumn{1}{|c|}{\textcolor{blue}{\textbf{CFS}}} & \multicolumn{1}{|c|}{\textbf{FCBF}} & \multicolumn{1}{|c|}{\textbf{FiR}}	&  \multicolumn{1}{|c|}{\textbf{MRMR}}	& \multicolumn{1}{|c|}{\textbf{Sem FS}} \\  \hline
		
    \multicolumn{1}{|l|}{Colon}       & 24.2  & 22.6  & 24.2 &  \textbf{19.4} & \textcolor{blue}{25.8} & 22.6 & \textbf{19.4} & 21.0 & 21.0   \\ \hline
	%  \multicolumn{1}{|l|}{SRBCT}               & \textbf{0.0}  & \textbf{0.0}   &  \textbf{0.0} &  \textbf{0.0}  & \textbf{0.0} & 4.8 & \textbf{0.0} & 4.8 & \textbf{0.0}  \\ \hline
	%  \multicolumn{1}{|l|}{PIE10P}      & \textbf{0.0} &  \textbf{0.0} & 0.5 & \textbf{0.0} & N/A & 1.0  & \textbf{0.0}  & 24.8 & \textbf{0.0} \\ \hline
		\multicolumn{1}{|l|}{\textcolor{blue}{Lymphoma}}  					& \textcolor{blue}{\textbf{2.2}}   & \textcolor{blue}{\textbf{2.2}}     & \textcolor{blue}{\textbf{2.2}}  &  \textcolor{blue}{\textbf{2.2}}  & \textcolor{blue}{N/A}   & \textcolor{blue}{3.3} & \textcolor{blue}{\textbf{2.2}}  & \textcolor{blue}{22.8} & \textcolor{blue}{\textbf{2.2}} \\ \hline
		\multicolumn{1}{|l|}{Leukemia1}  			  & 5.6   & \textbf{2.8}  & 6.9  &  6.9 & \textcolor{blue}{N/A}  & 5.6 & 4.2 & 9.7 & 5.6  \\ \hline		
		\multicolumn{1}{|l|}{B-Tumor1}  		  & 13.3  & 12.2  & 13.3 &  11.1 & \textcolor{blue}{N/A} & \textcolor{blue}{18.9} & 11.1 & \textcolor{blue}{25.6} & \textbf{10.0} \\ \hline
    \multicolumn{1}{|l|}{Leukemia}  					& \textbf{2.8}  & 12.5 & \textbf{2.8} & \textbf{2.8} & \textcolor{blue}{N/A} & 4.2 & 4.2 & 8.3 & \textbf{2.8} \\ \hline
		\multicolumn{1}{|l|}{Example1}  					& 2.3   & 2.2 & 2.2  & 3.7 & \textcolor{blue}{N/A} & 6.3 & \textbf{2.1} & 28.3 & 2.4 \\ \hline
  %	\multicolumn{1}{|l|}{ORL0P}      					& 4.0 &  5.0 & 2.0 & \textbf{1.0} & N/A & \textbf{1.0}  & 2.0  & 68.0  & \textbf{1.0}\\ \hline
		\multicolumn{1}{|l|}{B-Tumor2}  		  & 34.0  & \textbf{22.0}  & 30.0 & \textbf{22.0} & \textcolor{blue}{N/A} & \textcolor{blue}{36.0} & 24.0 & \textcolor{blue}{42.0} & 26.0 \\ \hline
	  \multicolumn{1}{|l|}{\textcolor{blue}{P-Tumor}}  		& \textcolor{blue}{7.8}   & \textcolor{blue}{5.9}   & \textcolor{blue}{\textbf{4.9}} & \textcolor{blue}{7.8}  & \textcolor{blue}{N/A} & \textcolor{blue}{9.8} & \textcolor{blue}{7.8} & \textcolor{blue}{12.7} & \textcolor{blue}{8.8}  \\ \hline
	  \multicolumn{1}{|l|}{L-Cancer}  				& 5.9   &  6.4  & 4.9  & 4.9   & \textcolor{blue}{N/A}    & 6.4 & 5.4 & 11.8 & 5.9 \\ \hline
	%	\multicolumn{1}{|l|}{SMK-187}  				& 41.7 & 40.6 & 53.5 & 24.6 & N/A & 33.2 & \textbf{23.5} & 33.2 & 24.1 \\ \hline
	  \multicolumn{1}{|l|}{Dexter}  					  & 6.7  & \textbf{6.0} & 7.7 & 9.3 & \textcolor{blue}{N/A} & 15.3 & 6.7 & 18.0 & 6.3 \\ \hline
	%	\multicolumn{1}{|l|}{GLI-85}  					  & 14.1 &  12.9 & 17.6 & \textbf{11.8} & N/A & 20.0 & 14.1 & 16.5 & 14.1 \\ \hline
 	  \multicolumn{1}{|l|}{\textcolor{blue}{Dorothea}}  					& \textcolor{blue}{\textbf{25.0}} &  \textcolor{blue}{26.0}  & \textcolor{blue}{\textbf{25.0}}  &   \textcolor{blue}{N/A} & \textcolor{blue}{N/A} & \textcolor{blue}{N/A} & \textcolor{blue}{\textbf{25.0}}   & \textcolor{blue}{N/A}  & \textcolor{blue}{\textbf{25.0}}   \\ \hline
	\end{tabular}}
\end{table}
\end{frame}


% Sec��o 7 - Conclus\~oes 
%
\section[Conclus\~oes]{Conclus\~oes}

%
% Slide 11
%
\begin{frame}{Conclus\~oes}
\begin{itemize}
	\vfill
	\item O projeto consistiu em....
	
	\vfill
	\item Atingiu-se uma solu\c{c}\~ao ...
	
	\vfill
	\item Os m\'etodos propostos foram avaliados:
	\begin{itemize}
		\item sobre dados de dom\'inio p\'ublico 
		\item comparativamente com m\'etodos existentes em implementa\c{c}\~oes de dom\'inio p\'ublico 
	\end{itemize}
	
	\vfill
	\item Os m\'etodos propostos complementam os existentes, podendo ser combinados entre si
	
\end{itemize}
\end{frame}

%
% Slide 12
%
\subsection[Trabalho Futuro]{Trabalho Futuro}
\begin{frame}{Trabalho Futuro}
Perspetivam-se as dire\c{c}\~oes de trabalho futuro:
\begin{itemize}
	\vfill
	\item Melhorar a proposta de ...
	
	\vfill
	\item Explorar o afinamento dos par\^ametros ...
	
	\vfill
	\item ..
	
	\vfill \hfill \footnotesize{Slides elaborados em \LaTeX \ com o package \textsc{beamer}}.
\end{itemize}
\end{frame}

\end{document}


